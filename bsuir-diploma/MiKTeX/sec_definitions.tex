\sectioncentered*{Определения и сокращения}

В настоящей пояснительной записке применяются следующие определения и сокращения.

\textit{Инициализация} -- приведение областей памяти в состояние, исходное для последующей обработки или размещения данных.

\textit{Программа} -- данные, предназначенные для управления конкретными компонентами системы обработки информации в целях реализации определенного алгоритма.

\textit{Программное обеспечение} -- программы, процедуры, правила и любая соответствующая документация, относящиеся к работе вычислительной системы.

\textit{Программирование} -- практическая деятельность по созданию про-грамм.

\textit{Программный модуль} -- программа или функционально завершенный фрагмент программы, предназначенный для хранения, трансляции, объединения с другими программными модулями и загрузки в оперативную память.

\textit{Подпрограмма} -- программа, являющаяся частью другой программы и удовлетворяющая требованиям языка программирования к структуре программы.

\textit{Спецификация программы} -- формализованное представление требований, предъявляемых к программе, которые должны быть удовлетворены при ее разработке, а также описание задачи, условия и эффекта действия без указания способа его достижения.

\textit{Веб-интерфейс} -- это совокупность средств, при помощи которых пользователь взаимодействует с веб-сайтом или любым другим приложением через браузер.

ООП -- объектно-ориентированное программирование

ПС -- программное средство

ОС -- операционная система

БД -- база данных

СУБД -- система управления базами данных

JPA -- спецификация Java EE, предоставляет возможность сохранять в удобном виде Java-объекты в базе данных

DAO -- data access object