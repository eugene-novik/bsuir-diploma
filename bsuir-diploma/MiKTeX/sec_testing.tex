\section{Тестирование приложения}
\label{sec:testing}

Для оценки правильности работы программного средства было проведено тестирование. Тест-кейсы для функционального требования <<Взаимодействие с пользователем>> представлены в таблице \ref{sec:testing:interaction_cases}



\begin{longtable}[l]{| >{\raggedright}m{0.3\textwidth}
                  | >{\raggedright}m{0.3\textwidth}
                  | >{\raggedright\arraybackslash}m{0.3\textwidth}|}
  \caption{Тестирование взаимодействия с пользователем}
  \label{sec:testing:interaction_cases} \tabularnewline

  \hline
       Название тест-кейса и его описание & Ожидаемый результат  & Полученный результат \\
   \hline
   \centering{1} & \centering{2} & \centering{3} \tabularnewline
   \hline
   Cоздание учетной записи пользователя \\ а) Ввести имя, почту, пароль. \\ б) нажать кнопку sign up  &
   a) Отображается сообщение об успешном создании пользователя \\ б) Происходит перенаправление на страницу входа в приложение &
   Пройден \\
   \hline


  Вход в приложение используя ранее созданную учетную запись пользователя\\
   а) Ввести почту и пароль \\
   б) Нажать кнопку sign in
   &
   а) Происходит перенаправление на страницу с объявлениями\\
   б) Отображается список существующих объявлений
   &
   Пройден \\
   \hline

   Востановление учетной записи пользователя \\
   a) Нажать на кнопку forgot password \\
   б) Ввести почту при создании пользователя \\
   в) Нажать кнопку restore password
   &
   а) На указанную почту приходит письмо с новом поролем \\
   б) Пользователь успешно входит в систему используя новый пароль\\
   в) Отображается список существующих объявлений 
   &
   Пройден \\
   % \pagebreak

   % \caption*{Продолжение таблицы~\ref{sec:testing:interaction_cases}} \\
   % \hline
   % 1 & 2 & 3 \\
   % \hline

  \pagebreak
  \caption*{Продолжение таблицы~\ref{sec:testing:interaction_cases}} \\
   \hline
   \centering 1 & \centering 2 & \centering 3 \tabularnewline
   \hline

   Добавления объявления \\
   a) Нажать кнопку create note  \\
   б) Заполнить обязательные поля \\
   в) Нажать кнопку add
   % \begin{enumerate}
   % \end{enumerate}
   &
   а) На вкладке My note отображается новое объявление\\
   б) Новое объявление должно быть доступно для поиска \\
   &
   Пройден\\
   \hline


   Изменить существующее объявление \\
   а) Выбрать объявление \\
   б) Нажать кнопку Edit \\
   в) Изменить несколько полей \\
   г) Нажать кнопку Save
   &
   а) Измененные данные сохранились в объявлении
   &
   Пройден \\
   \hline

   Удалить существующее объявление \\
   а) Выбрать нужное объявление \\
   б) Нажать кнопку delete
   &
   а) Объявление пропадает со вкладки My notes \\
   б) Объявление становиться недоступным для поиска 
   &
   Пройден \\
   \hline
   
   Добавить фотографию в объявление \\
   а) Выбрать объявление \\
   б) Нажать кнопку Add Picture \\
   в) Выбрать изображение на компьютере \\
   г) Нажать кнопку Upload
   &
   а) Новая фотография добавилась к выброному объявлению
   &
   Пройден \\
  \pagebreak
  \caption*{Продолжение таблицы~\ref{sec:testing:interaction_cases}} \\
   \hline
   \centering 1 & \centering 2 & \centering 3 \tabularnewline
   \hline
   Поиск объявлений \\
   а) Открыть страницу поиска \\
   б) Ввести необходимые критерии \\
   в) Нажать кнопку Search Notes
   &
   а) Найденные объявления отображаются в таблице, в центре экрана \\
   б) Найденные обявлнения можно просматривать в новой вкладе
   &
   Пройден \\
   \hline
   Оставить отзыв на объявление \\
   а) Выбрать необходимое объявление \\
   б) Ввести отзыв в поле Comment \\
   в) Нажать кнопку Add Comment
   &
   а) Написанный отзыв отображается снизу, под основной информацией объявления \\
   б) Отображается значок для удаления коментария
   &
   Пройден \\
   \hline
   
   Редактирование профиля \\
   а) Перейти в раздел Settings \\
   б) Надать кнопку Edit \\
   в) Изменить необходимые данные \\
   г) Нажать кнопку Save
   &
   а) Страница обновляется \\
   б) Новая информация отображается в профиле пользователя
   &
   Пройден \\
   \hline
\end{longtable}


Таким образом, результат тестирования подтверждает, что программное средство web прилложение для продажи квартир функционирует в полном соответствии со спецификацией требований.
  % \begin{longtable}[l]{| >{\raggedright}m{0.3\textwidth}
  %                 | >{\centering}m{0.3\textwidth}
  %                 | >{\centering}m{0.3\textwidth}|}
  % \caption{Тестирование анализирования входных файлов}
  % \label{sec:testing:analyzing_cases} \tabularnewline

  % \hline
  %      Название тест-кейса и его описание & Ожидаемый результат & Полученный результат
  %    \tabularnewline
  %  \hline
  %  1 & 2 & 3 \\
  %  \hline
  %  Запуск программы без аргументов
  %  \begin{enumerate}
  %  \item Ввести имя исполняемого модуля без аргументов
  %  \item нажать клавишу Ввода
  %  \end{enumerate}
  %  &
  %  \begin{enumerate}
  %  \item Имя исполняемого модуля отображается в консоли
  %  \item Отображается справочная информация об использовании приложения
  %  \end{enumerate}
  %  &
  %  \begin{enumerate}
  %  \item Имя исполняемого модуля отображается в консоли
  %  \item Отображается справочная информация об использовании приложения
  %  \end{enumerate} \\
  %  \hline
   %-----------------------------------------
  % \caption*{Продолжение таблицы~\ref{table:econ:calculation_cost_and_price}}
  % \hline
  %   {\begin{center}
  %      Наименование статей
  %   \end{center} } & \mbox{Норматив,} \% & Методика расчета & \mbox{Значение,} руб. \\
  %  \hline
   % Прогнозируемая цена без налогов
   % &
   % & $ \text{Ц}_{\text{п}} = \text{С}_{\text{п}} + \text{П}_{\text{с}}$
   % & \num{\estimatedPrice} \\
   % \hline
   % Отчисления и налоги в местный и республиканский бюджеты
   % & $ \text{Н}_{\text{мр}} = \num{\localRepubTaxNormative} $
   % & $ \text{О}_{\text{мр}} = { \text{Ц}_{\text{п}} \cdot \text{Н}_{\text{мр}} } / { \num{100} - \text{Н}_{\text{мр}} } $
   % & \num{\localRepubTax} \\
   % \hline
   % Налог на добавленную стоимость
   % & $ \text{Н}_{\text{дс}} = \num{\ndsNormative} $
   % & $ \text{НДС}_{\text{}} = { (\text{Ц}_{\text{п}} + \text{О}_{\text{мр}}) \cdot \text{Н}_{\text{дс}} } / \num{100} $
   % & \num{\nds} \\
   % \hline
   % Прогнозируемая отпускная цена
   % &
   % & $ \text{Ц}_{\text{о}} = \text{Ц}_{\text{п}} + \text{О}_{\text{мр}} + \text{НДС} $
   % & \num{\sellingPrice} \\
   % \hline
