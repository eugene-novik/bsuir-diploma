\sectioncentered*{Реферат}
\thispagestyle{empty}
%%
%% ВНИМАНИЕ: этот реферат не соответствует СТП-01 2013
%% пример оформления реферата смотрите здесь: http://www.bsuir.by/m/12_100229_1_91132.docx
%%


\begin{center}
Пояснительная записка \pageref*{LastPage}c., \totfig{}~рис., \tottab{}~табл., \toteq{}~формул и \totref{}~литературный источник.\\
ЗАПУТЫВАНИЕ, VHDL, ПАРСИНГ, ОБФУСКАЦИЯ, RTL
\end{center}

Предметной областью разработки является сфера защиты интеллектуальной собственности, анализа кодов и обфускации. Объект разработки -- приложение для конечного пользователя, предоставляющее функционал по анализу и запутыванию кода.

Целью разработки является создание удобного, простого приложения, пригодного для решения практических задач, возникающих при работе в области моделирования дизайнов на языке VHDL.

При разработке проекта использовалась среда разработки Sublime Text 3 с различными расширениями, такими как, LatexTools, Package Control и т.д. Язык программирования приложения -- Ruby.


Результатом разработки стало простое в использовании приложение, которое может быть легко интегрировано в процесс работы, предоставляющее различные возможности по проверке и запутыванию кода. Разработаны диаграмма компонентов, диаграмма потоков данных, а также различные схемы алгоритмов.

Предполагается использование приложения разработчиками цифровых микросхем.

Разработанное приложение является экономически эффективным, оно полностью  оправдывает средства, вложенные в его разработку.
% Дипломный проект выполнен на 6 листах формата А1 с пояснительной запиской на~\pageref*{LastPage} страницах, без приложений справочного или информационного характера.

% Целью дипломного проекта является разработка удобного в использовании инструмента, пригодного для решения практических задач, возникающих в реальных проектах, связанных с вероятностным моделированием.

% Для достижения цели дипломного проекта была разработана библиотека кода для \dotnet{}, предназначенная для представления и обучения структуры вероятностной сети по экспериментальным данным.
% Библиотека может быть использована в реальных проектах, использующих вероятностный подход к решению проблемы.
% В библиотеке реализовано несколько алгоритмов, имеющих различные качественные характеристики.

% В разделе технико"=экономического обоснования был произведён расчёт затрат на создание ПО, а также прибыли от разработки, получаемой разработчиком.
% Проведённые расчёты показали экономическую целесообразность проекта.

% Пояснительная записка включает раздел по охране труда, в котором была произведена оценка пожарной безопасности на предприятии, где частично разрабатывался данный дипломный проект.

\clearpage
